\chapter{Interval}


\section{Introduction}
\rih{Two-way range.} The current scanning node as the pivot, need to scan its left neighbors and right neighbors. 
$$
|\leftarrow p \rightarrow |
$$

If the relationship between the pivot and its neighbors is symmetric, since scanning range is $[i-k, i+k]$ and iterating from left to right, only consider $[i-k, i]$ to avoid duplication.
$$
|\leftarrow p
$$

\section{Operations}
\runinhead{Merge intervals.} Given a collection of intervals, merge all overlapping intervals.

\textbf{Core clues}:
\begin{enumerate}
\item Sort the intervals
\item When does the overlapping happens?
[0, 5) vs. [2, 6); [0, 5) vs. [2, 4)
\end{enumerate}
\begin{python}
def merge(self, itvls):
    if not itvls:
        return []

    itvls.sort(key=lambda x: x.start)
    ret = [itvls[0]]
    for cur in itvls[1:]:
        pre = ret[-1]
        if cur.start <= pre.end:  # overlap
            pre.end = max(pre.end, cur.end)
        else:
            ret.append(cur)

    return ret
\end{python}

\runinhead{Insert intervals.} Given a set of non-overlapping intervals, insert a new interval into the intervals (merge if necessary). Assume that the intervals were initially sorted according to their start times.

\textbf{Core clues}
\begin{enumerate}
\item Partition the original list of intervals to left-side intervals and right-side intervals according to the new interval. 
\item Merge the intermediate intervals with the new interval. Need to mathematically prove it works as expected.
\end{enumerate}

\begin{python}
def insert(self, itvls, newItvl):
    s, e = newItvl.start, newItvl.end
    left = filter(lambda x: x.end < s, itvls)
    right = filter(lambda x: x.start > e, itvls)
    if len(left)+len(right) != len(itvls):
        s = min(s, itvls[len(left)].start)
        e = max(e, itvls[-len(right)-1].end)

    return left + [Interval(s, e)] + right

\end{python}

\section{Event-driven algorithms}
\subsection{Introduction}
The core philosophy of event-driven algorithm:
\begin{enumerate}
\item \textbf{Event}: define \textit{event}; the event are sorted by time of appearance.
\item \textbf{Heap}: define \textit{heap meaning}.
\item \textbf{Transition}: define \textit{transition functions} among events impacting the.
heap. 
\end{enumerate} 

\subsection{Questions}
\runinhead{Maximal overlaps.} Given a list of number intervals, find max number of overlapping
intervals. 
\runinhead{Core clues:}
\begin{enumerate}
\item \textbf{Event}: Every new start of an interval is an event. Scan the sorted intervals (sort the interval by \textit{start}).
\item \textbf{Heap meaning}: Heap stores the \textit{end} of the interval. 
\item \textbf{Transition}: Put the ending time into heap, and pop the ending time earlier than the new start time from heap.
\end{enumerate}
\newpage
\begin{python}
def max_overlapping(intervals):
    maxa = 0
    intervals.sort(key=lambda x: x.start)
    h_end = []
    for itvl in intervals:
        heapq.heappush(end_heap, itvl.end)
        
        while h_end and h_end[0] <= itvl.start:
            heapq.heappop(h_end)

        maxa = max(maxa, len(h_end))

    return maxa
\end{python}

