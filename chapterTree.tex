\chapter{Tree}

\section{Binary Tree}
\subsection{Introductions}
\runinhead{Get parent ref.} To get a parent reference (implicitly), return the current recursion function to its parent to maintain the path. Sample code:
\begin{java}
Node deleteMin(Node x) {
    if (x.left == null) return x.right;
    x.left = deleteMin(x.left);
    x.count = 1+size(x.left)+size(x.right);
    return x;
}
\end{java}
\runinhead{Search.} To search a node in binary tree (not necessarily BST), use dfs:
\begin{python}
def dfs(self, root, t, path, found):
    if not root or found[0]:  # post-call check
        return

    path.append(root)
    if root == t:
        found[0] = True

    self.dfs(root.left, t, path, found)
    self.dfs(root.right, t, path, found)
    if not found[0]:
        path.pop()  # 1 pop() corresponds to 1 append()
\end{python}
The `found` is a wrapper for boolean to keep it referenced by all calling stack. 

\rih{Lowest common ancestor.} In BST, the searching is straightforward. In normal binary tree, construct the path from root to $node_1$ and $node_2$ respectively, and \textbf{diff} the two paths.

\rih{Find all paths.} Find all paths from root to leafs. For every currently visiting node, add itself to path; search left, search right and pop itself. Record current result when reaching the leaf.
\begin{python}
def dfs_path(self, cur, path, ret):
    if not cur:
        return

    path.append(cur)
    if not cur.left and not cur.right:
        ret.append("->".join(map(lambda x: str(x.val), path)))

    self.dfs_path(cur.left, path, ret)
    self.dfs_path(cur.right, path, ret)
    path.pop()
\end{python}
 
\section{Binary Search Tree (BST)}
\runinhead{Array and BST.}Given either the \textbf{preorder} or \textbf{postorder} (but not inorder) traversal of a BST containing N distinct keys, it is possible to reconstruct the shape of the BST. 
\subsection{Operations}
\runinhead{Rank.}

\subsection{Range search}
\runinhead{1-d range count}
\begin{java}
int size(Key lo, Key hi) {
    if (contains(hi)) return rank(hi)-rank(lo)+1;
    else              return rank(hi)-rank(lo);
}
\end{java}
\runinhead{1-d range search}
TODO

\runinhead{Closest value} Find the value in BST that is closet to the \pyinline{target}.

Clues:
\begin{enumerate}
\item Find the value just $\leq$ the target.
\item Find the value just $\geq$ the target.
\end{enumerate}

Code for finding either the lower value or higher value:
\begin{python}
def find(self, root, target, ret, lower=True):
  if not root:
    return

  if root.val == target:
    ret[0] = root.val
    return

  if root.val < target:
    if lower:
      ret[0] = max(ret[0], root.val)

    self.find(root.right, target, ret, lower)
  else:
    if not lower:
      ret[0] = min(ret[0], root.val)

    self.find(root.left, target, ret, lower)
\end{python}

\runinhead{Closet values} Find $k$ values in BST that are closet to the \pyinline{target}.

Clues:
\begin{enumerate}
\item Find the predecessors $\triangleq \{node | node.value \leq target\}$.
\item Find the successors $\triangleq \{node | node.value \geq target\}$.
\item Merge the predecessors and successors as in merge in MergeSort to get he $k$ values. 
\end{enumerate}

Code for finding the predecessors:
\begin{python}
def predecessors(self, root, target, stk):
  if not root:
    return

  self.predecessors(root.left, target, stk)
  if root.val <= target:
    stk.append(root.val)
    self.predecessors(root.right, target, stk)
\end{python}

\section{Interval Search Tree}
TODO

\section{Segment Tree}
\subsection{Introduction}
The structure of \textbf{Segment Tree} is a binary tree which each node has two attributes start and end denote an segment/interval. Notice that by practice, the interval is normally $[start, end)$ but sometimes it can be $[start, end]$, which depends on the question definition. 

Structure:  
\begin{lstlisting}[columns=flexible]
                     [0, 4, count=3]
                     /             \
          [0,2,count=1]             [2,4,count=2]
          /         \               /            \
   [0,1,count=1] [1,2,count=0] [2,3,count=1], [3,4,count=1]
\end{lstlisting}

Variants:
\begin{enumerate}
\item Sum Segment Tree.
\item Min/Max Segment Tree.
\item Count Segment Tree. 
\end{enumerate}

For a \textbf{Maximum Segment Tree}, which each node has an extra value max
to store the maximum value in this node's interval.

\subsection{Operations}
Components in Segment Tree:
\begin{enumerate}
\item Build
\item Query 
\item Modify 
\end{enumerate}

Notice:
\begin{enumerate}
\item Only build need to change the start and end recursively.
\item Pre-check is preferred in recursive calls.
\end{enumerate}


Code:
\begin{python}
DEFAULT = 0
f = lambda x, y: x+y


class Node(object):
    def __init__(self, start, end, m):
        self.start, self.end, self.m = start, end, m
        self.left, self.right = None, None


class SegmentTree(object):
    def __init__(self, A):
        self.A = A
        self.root = self.build_tree(0, len(self.A))

    def build_tree(self, s, e):
        """
        segment: [s, e)
        """
        if s >= e:
            return None

        if s+1 == e:
            return Node(s, e, self.A[s])

        left = self.build_tree(s, (s+e)/2)
        right = self.build_tree((s+e)/2, e)
        val = DEFAULT
        if left: val = f(val, left.m)
        if right: val = f(val, right.m)
        root = Node(s, e, val)
        root.left = left
        root.right = right

        return root

    def query(self, root, s, e):
        """
        :type root: Node
        """
        if not root:
            return DEFAULT

        if s <= root.start and e >= root.end:
            return root.m

        if s >= root.end or e <= root.start:
            return DEFAULT

        l = self.query(root.left, s, e)
        r = self.query(root.right, s, e)
        return f(l, r)

    def modify(self, root, idx, val):
        """
        :type root: Node
        """
        if not root or idx >= root.end or idx < root.start:
            return

        if idx == root.start and idx == root.end-1:
            root.m = val
            self.A[idx] = val
            return

        self.modify(root.left, idx, val)
        self.modify(root.right, idx, val)
        val = DEFAULT
        if root.left: val = f(val, root.left.m)
        if root.right: val = f(val, root.right.m)
        root.m = val
\end{python}
