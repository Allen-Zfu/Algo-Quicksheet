\chapter{Dynamic Programming}



\section{Introduction}
The core philosophy of dp:
\begin{enumerate}
\item The definition of \textbf{states} 
\item The definition of the \textbf{transition functions} among states 
\end{enumerate} 

The so called concept dp as memoization of recursion does not grasp the core philosophy of dp. 

The formula in the following section are unimportant. What is important is the definition of dp array and transition function derivation.
\subsection{Common practice}
\begin{enumerate}
\item Use $n+1$ dp arrays to reserve space for dummies. 
\item Iteration range is $[1, n+1)$.
\item $n+k$ for k dummies  
\end{enumerate}
\subsection{Space optimization}
To avoid MLE, we need to carry out space optimization. Let $o$ be other subscripts, $f$ be the transition function. 

Firstly,
$$
F_{i, o} = f\big(F_{i-1, o'}\big)
$$
should be reduced to 
$$
F_{o} = f\big(F_{o'}\big)
$$

Secondly,
$$
F_{i, o} = f\big(F_{i-1, o'}, F_{i-2. o'}\big)
$$
should be reduced to 
$$
F_{i, o} = f\big(F_{(i-1)\%2, o'}, F_{(i-2)\%2. o'}\big)
$$

More generally, we can be $(i-b)\%a$ to reduce the space down to $a$.

Notice,
\begin{enumerate}
\item Must iterate $o$ \textbf{backward} to un-updated value. 
\end{enumerate}



\section{Sequence}
Longest common subsequence. Let $F_{i, j}$ be the LCS at string $a[:i]$ and $b[:j]$. We have two situations: $a[i]==b[j]$ or not.
\begin{eqnarray*}
F_{i. j} = \left\{ \begin{array}{rl}
  F_{i-1, j-1}+1, &\mbox{ if $a[i]==b[j]$} \\
  \max(F_{i-1, j}, F_{i,j-1}),&\mbox{ otherwise}
       \end{array} \right.
\end{eqnarray*}
\\
Longest common substring. Let $F_{i, j}$ be the LCS at string $a[:i]$ and
$b[:j]$. We have two situations: $a[i]==b[j]$ or not.
\begin{eqnarray*}
F_{i. j} = \left\{ \begin{array}{rl}
  F_{i-1, j-1}+1, &\mbox{ if $a[i]==b[j]$} \\
  0, &\mbox{ otherwise}
       \end{array} \right.
\end{eqnarray*}

Because it is not necessary that $F_{i,j}\geq F_{i',j'}, \forall i,j\cdot i>i', j>j'$, the $gmax=max\big(\{{F_{i,j}\}\big)$.

\section{Backpack}
Given $n$ items with weight $w_i$ and value $v_i$, an integer $C$ denotes the size of a backpack. What is the max value you can fill this backpack?

Let $F_{i, c}$ be the max size we can carry for index $0..i$ with capacity $c$. We have 2 choices: take the $i$-th item or not.
\begin{eqnarray*}
F_{i, c}= \max\big(&&F_{i-1, c}, \\
&&F_{i-1, c-w_i}+v_i\big)
\end{eqnarray*}


\section{Local and global extremes}
\subsection{Long and short stocks}
The following formula derives from the question: Best Time to Buy and Sell Stock IV. 

Let $local_{i, j}$ be the max profit at day $i$ with $j$ transactions with last transactions ENDED at day $i$. Let $global_{i, j}$ be the max profit at day $i$ with $j$ transactions.  
\begin{eqnarray}
&& local_{i,j} = \max\Big(global_{i-1.j-1}+\Delta, local_{i-1,j}+\Delta\Big) \nonumber \\
&& global_{i,j} = \max\Big(local_{i, j}, global_{i-1,j}\Big)
\end{eqnarray}

Notice:
\begin{enumerate}
\item Consider opportunity costs.
\item The global min is not $glocal[-1]$ but $\max\big(\{global[i]\}\big)$.
\item You must sell the stock before you buy again (i.e. you can not have higher than 1 in stock position). 
\end{enumerate}

\subsubsection{Space optimization}
\begin{eqnarray}
&& local_{j} = \max\Big(global_{j-1} + \Delta, local_{j}+\Delta\Big)
\nonumber \\
&& global_{j} = \max\Big(local_{j}, global_{j}\Big)
\end{eqnarray}

Notice,
\begin{enumerate}
\item Must iterate $j$ \textbf{backward}; otherwise we will use the updated value. 
\end{enumerate}

\subsubsection{Alternative definitions}
Other possible definitions: let $global_{i, j}$ be the max profit
at day $i$ with UP\ TO $j$ transactions. Then, 
\begin{eqnarray}
&& local_{i,j} = \max\Big(global_{i-1.j-1} + \max(0, \Delta), local_{i-1,j}+\Delta\Big)
\nonumber \\
&& global_{i,j} = \max\Big(local_{i, j}, global_{i-1,j}\Big)
\end{eqnarray}
and $global[-1]$ is the global max. 

\section{Game theory - multi players}
Assumption: the opponent take the optimal strategy for herself. 

\subsection{Coin game}
\runinhead{Same side} There are $n$ coins with different value in a line. Two players take turns to take 1 or 2 coins from left side. The player who take the coins with the most value wins.

let $F_i^p$ represents maximum values he can get for index $i..last$, for the person p. There are 2 choices: take the $i$-th coin or take the $i$-th and $(i+1)$-th coin.
\begin{eqnarray*}
F_i^p = \max\big(&A_i&+S[i+1:]-F_{i+1}^{p'},  \\
&A_i&+A_{i+1}+S[i+2:]-F_{i+2}^{p'}\big)
\end{eqnarray*}
The above equation can be further optimized by merging the sum $S$.

\runinhead{Dual sides}There are n coins in a line. Two players take turns to take a coin from one of the ends of the line until there are no more coins left. The player with the larger amount of money wins.

let $F_{i, j}^p$ represents maximum values he can get for index $i..j$, for
the person p. There are 2 choices: take the $i$-th coin or take the $j$-th coin.
\begin{eqnarray*}
F_{i,j}^p = \max\big(&A_i&+S[i+1:j]-F_{i+1,j}^{p'},  \\
&A_j&+S[i:j-1]-F_{i,j-1}^{p'}\big)
\end{eqnarray*}
