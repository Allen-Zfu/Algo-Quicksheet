\chapter{Arithmetic}


\section{Big Number}
\rih{Plus One.} Given a non-negative number represented as an array of digits, plus one to the number.
\begin{python}
def plusOne(self, digits):
    for i in xrange(len(digits)-1, -1, -1):
        digits[i] += 1
        if digits[i] < 10:
            return digits
        else:
            digits[i] -= 10

    # if not return within the loop 
    digits.insert(0, 1)
    return digits
\end{python}

\section{Polish Notations}
Polish Notation is in-fix while Reverse Polish Notation is post-fix. 

Reverse Polish notation (RPN) is a mathematical notation in which every operator follows all of its operands (i.e. operands are followed by operators). RPN should be treated as the orthogonal expression.  

Polish notation (PN) is a mathematical notation in which every operator is followed by its operands. 

\subsection{Convert in-fix to post-fix (RPN)}
\pyinline{ret} stores the final result of reverse polish notation. \pyinline{stk} stores
the temporary result in strictly increasing order. 

In-fix
\begin{python}
5 + ((1 + 2) * 4) - 3
\end{python}

can be written as
\begin{python}
5 1 2 + 4 * + 3 - 
\end{python}
Core clues:
\begin{enumerate}
\item \rih{Stack}. The stack temporarily stores the operators of \textit{strictly increasing precedence order}, except for brackets, which are put onto stack directly.
\item \rih{Precedence}. Digits have the highest precedence, followed by \pyinline{*, /, +, (}. Notice that \pyinline{(} operator itself has the \textit{lowest} precedence.
\item \rih{Bracket}. \textit{Match} the brackets. 
\end{enumerate}
Code:
\begin{python}
def infix2postfix(self, lst):
  stk = []
  ret = []  # post fix result
  for elt in lst:
    if elt.isdigit():
      ret.append(elt)
    elif elt == "(":
      stk.append(elt)
    elif elt == ")":
      while stk and stk[-1] != "(":
        ret.append(stk.pop())
      stk.pop()  # pop "("
    else:
      # maintain invariant
      while stk and not precdn(stk[-1]) < precdn(elt):
        ret.append(stk.pop())
      stk.append(elt)

  while stk:  # clean up 
    ret.append(stk.pop())

  return ret
\end{python}

\subsection{Evaluate post-fix expressions}\label{section:evaluationPostFix}
Consider: 

In-fix
\begin{python}
5 + ((1 + 2) * 4) - 3
\end{python}

Post-fix
\begin{python}
5 1 2 + 4 * + 3 - 
\end{python}
Straightforward: use a \textit{stack} to store the number. Iterate the input, push
stack when hit numbers, pop stack when hit operators.

\subsection{Convert in-fix to pre-fix (PN)}
PN is the \textit{reverse} of RPN, thus, scan the expression from right to left; and \pyinline{stk} stores the temporary result in \textit{non-decreasing} order, except for brackets.


In-fix
\begin{python}
5 + ((1 + 2) * 4) - 3
\end{python}

can be written as the intermediate representation (IR)
\begin{python}
3 4 2 1 + * 5 + -
\end{python}

reverse as the pre-fix
\begin{python}
- + 5 * + 1 2 4 3
\end{python}

\begin{python}
  def infix2prefix(self, lst):
    """starting from right the left"""
    stk = []
    pre = []
    for elt in reversed(lst):
      if elt.isdigit():
        pre.append(elt)
      elif elt == ")":
        stk.append(elt)
      elif elt == "(":
        while stk and stk[-1] != ")":
          pre.append(stk.pop())
        stk.pop()
      else:
        # maintain invariant
        while stk and not precdn(stk[-1]) <= precdn(elt):  
          pre.append(stk.pop())
        stk.append(elt)

    while stk:
      pre.append(stk.pop())

    pre.reverse()
    return pre
\end{python}


\subsection{Evaluate pre-fix (PN) expressions}
Consider: 

In-fix
\begin{python}
5 + ((1 + 2) * 4) - 3
\end{python}

Pre-fix
\begin{python}
- + 5 * + 1 2 4 3
\end{python}

reverse as the intermediate representation (IR)
\begin{python}
3 4 2 1 + * 5 + -
\end{python}
Put into \textit{stack}, similar to evaluating post-fix \ref{section:evaluationPostFix}, but pay attention to operands order, which should be reversed when hitting a operator. 
