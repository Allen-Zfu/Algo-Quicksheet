\chapter{Arithmetic}
\section{Backtracking}
\subsection{Unidirection}
\rih{Insert operators.} Given a string that contains only digits 0-9 and a target value, return all possibilities to add binary operators (not unary) +, -, or * between the digits so they evaluate to the target value.

Example: 
\begin{align*}
``123", 6 \rightarrow [``1+2+3", ``1*2*3"] \\ 
``232", 8 \rightarrow [``2*3+2", ``2+3*2"] \\
\end{align*}
Clues:
\begin{enumerate}
\item Backtracking with \textit{horizontal} expanding
\item Special handling for multiplication - caching the expression \textit{predecessor} for multiplication association. 
\item Detect \textit{invalid} number with leading 0's
\end{enumerate}

\begin{python}
def addOperators(self, num, target):
  ret = []
  self.dfs(num, target, 0, "", 0, 0, ret)
  return ret

def dfs(self, num, target, pos, 
        cur_str, cur_val, 
        mul, ret
):
  if pos >= len(num):
    if cur_val == target:
      ret.append(cur_str)
  else:
    for i in xrange(pos, len(num)):
      if i != pos and num[pos] == '0':
        continue
        
      nxt_val = int(num[pos:i+1])
      if not cur_str:  # 1st number
        self.dfs(num, target, i+1, 
            "%d"%nxt_val, nxt_val,
            nxt_val, ret)
      else:  # +, -, *
        self.dfs(num, target, i+1, 
            cur_str+"+%d"%nxt_val, cur_val+nxt_val, 
            nxt_val, ret)
        self.dfs(num, target, i+1, 
            cur_str+"-%d"%nxt_val, cur_val-nxt_val, 
            -nxt_val, ret)
        self.dfs(num, target, i+1, 
    cur_str+"*%d"%nxt_val, cur_val-mul+mul*nxt_val, 
            mul*nxt_val, ret)
\end{python}
\subsection{Bidirection}
\rih{Insert parenthesis.} Given a string of numbers and operators, return all possible results from computing all the different possible ways to group numbers and operators. The valid operators are +, - and *.

Examples:
\begin{align*}
(2*(3-(4*5))) &= -34 \\
((2*3)-(4*5)) &= -14 \\
((2*(3-4))*5) &= -10 \\
(2*((3-4)*5)) &= -10 \\
(((2*3)-4)*5) &= 10
\end{align*}
Clues: Iterate the operators, divide and conquer - left parts and right parts and then combine result. \\
Code:
\begin{python}
def dfs_eval(self, nums, ops):
    ret = []
    if not ops:
        assert len(nums) == 1
        return nums

    for i, op in enumerate(ops):
        left_vals = self.dfs_eval(nums[:i+1], ops[:i])
        right_vals = self.dfs_eval(nums[i+1:], ops[i+1:])
        for l in left_vals:
            for r in right_vals:
                ret.append(self._eval(l, r, op))

    return ret
\end{python}

\section{Big Number}
\rih{Plus One.} Given a non-negative number represented as an array of digits, plus one to the number.
\begin{python}
def plusOne(self, digits):
    for i in xrange(len(digits)-1, -1, -1):
        digits[i] += 1
        if digits[i] < 10:
            return digits
        else:
            digits[i] -= 10

    # if not return within the loop 
    digits.insert(0, 1)
    return digits
\end{python}

\section{Polish Notation}
Polish Notation is in-fix while Reverse Polish Notation is post-fix. 

Reverse Polish notation (RPN) is a mathematical notation in which every operator follows all of its operands (i.e. operands are followed by operators).

Polish notation (PN) is a mathematical notation in which every operator is followed by its operands. 

\subsection{Convert In-fix to Post-fix}
\pyinline{ret} stores the final result of reverse polish notation. \pyinline{stk} stores
the temporary result in strictly increasing order. 

In-fix
\begin{python}
5 + ((1 + 2) * 4) - 3
\end{python}

can be written as
\begin{python}
5 1 2 + 4 * + 3 - 
\end{python}
Core clues:
\begin{enumerate}
\item \rih{Stack}. The stack temporarily stores the operators of \textit{strictly increasing precedence order}.
\item \rih{Precedence}. Digits have the highest precedence, followed by $*, /, +, ($. Notice that $($ operator itself has the \textit{lowest} precedence.
\item \rih{Bracket}. \textit{Match} the brackets. 
\end{enumerate}
Code:
\begin{python}
def infix2postfix(self, lst):
  stk = []
  ret = []  # post fix result
  for elt in lst:
    if elt.isdigit():
      ret.append(elt)
    elif elt == "(":
      stk.append(elt)
    elif elt == ")":
      while stk and stk[-1] != "(":
        ret.append(stk.pop())
      stk.pop()  # pop "("
    else:
      while stk and precdn(elt) <= precdn(stk[-1]):
        ret.append(stk.pop())
      stk.append(elt)

  while stk:  # clean up 
    ret.append(stk.pop())

  return ret
\end{python}

\subsection{Evaluate Post-fix Expressions}
\begin{python}
5 1 2 + 4 * + 3 - 
\end{python}
Straightforward: use a \textit{stack} to store the number. Iterate the input, push
stack when hit numbers, pop stack when hit operators.

\subsection{Convert In-fix to Pre-fix}
PN is the \textit{reverse} of RPN, thus, scan the expression from right to left; and \pyinline{stk} stores the temporary result in non-decreasing order. 


In-fix
\begin{python}
5 + ((1 + 2) * 4) - 3
\end{python}

can be written as
\begin{python}
3 4 2 1 + * 5 + - m
\end{python}

reverse as 
\begin{python}
- + 5 * + 1 2 4 3
\end{python}

\begin{python}
  def infix2prefix(self, lst):
    """starting from right the left"""
    stk = []
    pre = []
    for elt in reversed(lst):
      if elt.isdigit():
        pre.append(elt)
      elif elt == ")":
        stk.append(elt)
      elif elt == "(":
        while stk and stk[-1] != ")":
          pre.append(stk.pop())
        stk.pop()
      else:
        # < rather than <=
        while stk and precdn(elt) < precdn(stk[-1]):  
          pre.append(stk.pop())
        stk.append(elt)

    while stk:
      pre.append(stk.pop())

    pre.reverse()
    return pre
\end{python}


\subsection{Evaluate Pre-fix Expressions}
\begin{python}
- + 5 * + 1 2 4 3
\end{python}
