\chapter{Math}


\section{Prime Numbers}
\subsection{Sieve of Eratosthenes}
\subsubsection{Basics}
To find all the prime numbers less than or equal to a given integer n by Eratosthenes' method:
\begin{enumerate}
\item Create a list of consecutive integers from 2 through n: (2, 3, 4, ..., n).
\item Initially, let $p$ equal 2, the first prime number.
\item Starting from $p$, enumerate its multiples by counting to n in increments of $p$, and mark them in the list (these will be $2p$, $3p$, $4p$, ... ; the $p$ itself should not be marked).
\item Find the first number greater than $p$ in the list that is not marked. If there was no such number, stop. Otherwise, let $p$ now equal this new number (which is the next prime), and repeat from step 3.
\end{enumerate}

When the algorithm terminates, the numbers remaining not marked in the list are all the primes below $n$.

\subsubsection{Refinements}
The main idea here is that every value for $p$ is prime, because we have already marked all the multiples of the numbers less than $p$. Note that some of the numbers being marked may have already been marked earlier (e.g., 15 will be marked both for 3 and 5).

As a refinement, it is sufficient to mark the numbers in step 3 starting from $p^2$, as all the smaller multiples of $p$ will have already been marked at that point. This means that the algorithm is allowed to terminate in step 4 when $p^2$ is greater than n.

Another refinement is to initially list odd numbers only, (3, 5, ..., n), and count in increments of 2p in step 3, thus marking only odd multiples of $p$. This actually appears in the original algorithm. This can be generalized with wheel factorization, forming the initial list only from numbers coprime with the first few primes and not just from odds (i.e., numbers coprime with 2), and counting in the correspondingly adjusted increments so that only such multiples of $p$ are generated that are coprime with those small primes, in the first place.

\subsubsection{code}
\begin{lstlisting}[language=Python]
def countPrimes(n):
    """
    Find prime using Sieve's algorithm
    :type n: int
    :rtype: int
    """
    if n < 3:
        return 0

    is_prime = [True for _ in xrange(n)]
    is_prime[0], is_prime[1] = False, False
    for i in xrange(2, int(math.sqrt(n))+1):
        if is_prime[i]:
            for j in xrange(i*i, n, i):
                is_prime[j] = False

    return is_prime.count(True)
\end{lstlisting}
\subsection{TODO}


