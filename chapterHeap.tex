\chapter{Heap}


\section{Introduction}
Queue, Stack

\section{Python heapq}
Python only has built in min-heap. To use max-heap, you can: 
\begin{enumerate}
\item Revert the number: 1 becomes -1.
\item Wrap the data into another class and override \textbf{comparators}: \_\_cmp\_\_ or \_\_lt\_\_
\end{enumerate}

The following code presents the wrapping method:
\begin{lstlisting}[language=python]
class Value(object):
    def __init__(self, val):
        self.val = val
        self.deleted = False  # lazy delete 

    def __cmp__(self, other):
        # Reverse order by height to get max-heap
        assert isinstance(other, Value)
        return other.val - self.val
\end{lstlisting}

Normally the deletion by value in Python is $O(n)$, to achieve $O(\lg n)$ we can use \textbf{lazy deletion}. Before take the top of the heap, we do the following:
\begin{lstlisting}[language=python]
while heap and heap[0].deleted:
    heapq.heappop(heap)
\end{lstlisting}
\subsection{Summarizing properties}


