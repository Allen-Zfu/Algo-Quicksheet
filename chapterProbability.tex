\chapter{Probability}


\section{Shuffle}
Equal probability shuffle algorithm.

\subsection{Incorrect naive solution}
Swap current card $A_i$ with a random card from the entire deck $A$. 
\begin{java}
for (int i = 0; i < N; i++) {
   int j = (int) Math.random()*N;
   swap(a[i], a[j]);
}
\end{java}
\begin{python}
def shuffle(A):
  n = len(A)
  for i in xrange(n):
    j = random.randrange(n)
    A[i], A[j] = A[j], A[i]
\end{python}
Consider 3 cards, the easiest proof that this algorithm does not produce a uniformly random permutation is that it generates $3^3=27$ possible plans (consider steps in plans, duplicated result included), but there are only 3! = 6 permutations. Since $27\%3 \neq 0$, there must be some permutation is that is picked too much, and some that is picked too little.
\subsection{Knuth Shuffle}
Knuth (aka Fisher-Yates) shuffling algorithm guarantees to rearrange the elements in uniformly random order. 
\\
Core clues:
\begin{enumerate}
\item choose index uniformly $\in [i, N)$.
\item just like shuffling the poker card.
\end{enumerate}
\begin{java}
public void shuffle(Object[] a) {
    int N = a.length;
    for (int i = 0; i < N; i++) {
        // choose index uniformly in [i, N)
        int j = i + (int) (Math.random() * (N - i));
        swap(a[i], a[j]);
    }
}
\end{java}

\begin{python}
def shuffle(A):
  n = len(A)
  for i in xrange(n):
    j = random.randrange(i, n)
    A[i], A[j] = A[j], A[i]
\end{python}

\runinhead{Proof:}

$n!$ permutations. 

\subsection{Random Maximum}
Find the index of maximum number in an array with a probability of $\frac{1}{\#maxima}$.


\runinhead{Code:}
\begin{python}
def random_max(A):
    maxa, maxa_idx, k = A[0], 0, 1
    for i in xrange(1, len(A)):
        if maxa < A[i]:
            maxa, maxa_idx, k = A[i], i, 1
        elif maxa == A[i]:
            k += 1 
            if random.random() < float(1)/k:
                maxa_idx = i

    return maxa_idx

\end{python}
 
\runinhead{Proof:} Prove via mathematical induction. 

That is, assuming it works for any array of size $N$, prove it works for any array of size $N+1$.

So, given an array of size $N+1$, think of it as a sub-array of size $N$ followed a new element at the end. By assumption, your algorithm uniformly selects one of the max elements of the sub-array... And then it behaves as follows:

If the new element is larger than the max of the sub-array, return that element. This is obviously correct.

If the new element is less than the max of the sub-array, return the result of the algorithm on the sub-array. Also obviously correct.

The only slightly tricky part is when the new element equals the max element of the sub-array. In this case, let the number of max elements in the sub-array be $k$. Then, by hypothesis, your algorithm selected one of them with probability $\frac{1}{k}$. By keeping that same element with probability $\frac{k}{k+1}$, you make the overall probability of selecting that same element equal $$\frac{1}{k} \cdot \frac{k}{k+1} = \frac{1}{k+1}$$, as desired. You also select the last element with the same probability.

To complete the inductive proof, just verify the algorithm works on an array of size 1

\section{Sampling}
\subsection{Reservoir Sampling}
Sample $k$ from $A$, where the length of $A$ is either very large or unknown or dynamic. 
\begin{python}
def reservoir_sample(A, sz):
  R = A[:sz]  # sz for size
  
  for i in xrange(sz, len(A)):
    rv = random.randrange(0, i+1)
    if rv < sz: # rv for random variable
      R[rv] = A[i]
\end{python}

Until index $i$, with elements of length $i+1$ scanned, every element has a probability of $\frac{sz}{i+1}$ in the reservoir. 

\runinhead{Random pick index.} Given an array of integers with possible duplicates, randomly output the index of a given target number.
\begin{python}
def pick(self, target):
    sz = 0
    ret = None
    for idx, val in enumerate(A):
        if val == target:
            sz += 1
            rv = random.randrange(0, sz)
            if rv == 0:
                ret = idx

    return ret
\end{python}

\section{Distribution}
\subsection{Geometric Distr}
$$
P(X=k) = (1-p)^{k-1}p
$$

Expected number of trials of get a specific outcome:
$$
E[T] = \frac{1}{p}
$$
, which is the mean of the Geometric Distr. 
\subsection{Binomial Distr}
Notations:
$$
B(n, p)
$$
pmf:
$$
{n \choose k}\,p^{k}(1-p)^{n-k}
$$

\section{Expected Value}
\subsection{Dice value}
Expected value of rolling dice until getting a 3
\subsection{Coupon collector's problem}
Given n coupons, how many coupons do you expect you need to draw with replacement before having drawn each coupon at least once?

$$
E[T] = \Theta(n \lg n)
$$
, where $T$ is number of trial (i.e. time).

Let $T$ be the time to collect all. $t_i$ be the time to collect the $i$-th new different coupon. $p_i$ be the probability of collecting the $i$-th coupon after $i-1$ coupons have been collected. Observe that:
\begin{align*}
p_1 &= \frac{n}{n} \\ 
p_2 &= \frac{n-1}{n} \\
p_i &= \frac{n-i+1}{n}
\end{align*}
Thus,

\begin{align*}
E[T] &= \sum_{i=1}^n E[t_i] \\
&= \sum \frac{1}{p_i} \\
&= n(\frac{1}{1}+\frac{1}{2}+...+\frac{1}{n})
\end{align*}

\runinhead{Dice.} How many times must you roll a die until each side has appeared?
