\chapter{Probability}


\section{Shuffle}
Equal probability shuffle algorithm.

\subsection{Incorrect naive solution}
Swap current card $A_i$ with a random card from the entire deck $A$. 
\begin{java}
for (int i = 0; i < N; i++) {
   int j = (int) Math.random()*N;
   swap(a[i], a[j]);
}
\end{java}
\begin{python}
def shuffle(A):
  n = len(A)
  for i in xrange(n):
    j = random.randrange(n)
    A[i], A[j] = A[j], A[i]
\end{python}
Consider 3 cards, the easiest proof that this algorithm does not produce a uniformly random permutation is that it generates $3^3=27$ possible plans (consider steps in plans, duplicated result included), but there are only 3! = 6 permutations. Since $27\%3 \neq 0$, there must be some permutation is that is picked too much, and some that is picked too little.
\subsection{Knuth Shuffle}
Knuth (aka Fisher-Yates) shuffling algorithm guarantees to rearrange the elements in uniformly random order. 
\\
Core clues:
\begin{enumerate}
\item choose index uniformly $\in [i, N)$
\end{enumerate}
\begin{java}
public void shuffle(Object[] a) {
    int N = a.length;
    for (int i = 0; i < N; i++) {
        // choose index uniformly in [i, N)
        int j = i + (int) (Math.random() * (N - i));
        swap(a[i], a[j]);
    }
}
\end{java}

\begin{python}
def shuffle(A):
  n = len(A)
  for i in xrange(n):
    j = random.randrange(i, n)
    A[i], A[j] = A[j], A[i]
\end{python}
\section{Distribution}
\subsection{Geometric Distr}
$$
P(X=k) = (1-p)^{k-1}p
$$

Expected number of trials of get a specific outcome:
$$
E[T] = \frac{1}{p}
$$
, which is the mean of the Geometric Distr. 
\subsection{Binomial Distr}
Notations:
$$
B(n, p)
$$
pmf:
$$
{n \choose k}\,p^{k}(1-p)^{n-k}
$$

\section{Expected Value}
\subsection{Dice value}
Expected value of rolling dice until getting a 3
\subsection{Coupon collector's problem}
Given n coupons, how many coupons do you expect you need to draw with replacement before having drawn each coupon at least once?

$$
E[T] = \Theta(n \lg n)
$$
, where $T$ is number of trial (i.e. time).

Let $T$ be the time to collect all. $t_i$ be the time to collect the $i$-th new different coupon. $p_i$ be the probability of collecting the $i$-th coupon after $i-1$ coupons have been collected. Observe that:
\begin{align*}
p_1 &= \frac{n}{n} \\ 
p_2 &= \frac{n-1}{n} \\
p_i &= \frac{n-i+1}{n}
\end{align*}
Thus,

\begin{align*}
E[T] &= \sum_{i=1}^n E[t_i] \\
&= \sum \frac{1}{p_i} \\
&= n(\frac{1}{1}+\frac{1}{2}+...+\frac{1}{n})
\end{align*}

\runinhead{Dice.} How many times must you roll a die until each side has appeared?
