\chapter{Memory Complexity}

\section{Introduction}

\subsection{Tables}
The memory usage is based on Java.\\

\begin{tabular}{ll}
\hline\noalign{\smallskip}
\textbf{Type} & \textbf{Bytes} \\
\noalign{\smallskip}\hline\noalign{\smallskip}

boolean & 1 \\
byte & 1 \\
char & 2 \\
int & 4 \\
float & 4 \\
long & 8 \\
double & 8\\

\noalign{\smallskip}\hline\noalign{\smallskip}
\caption {for primitive types}
\end{tabular}

\begin{tabular}{ll}
\hline\noalign{\smallskip}
\textbf{Type} & \textbf{Bytes} \\
\noalign{\smallskip}\hline\noalign{\smallskip}

char[] & 2N+24 \\
int[] & 4N+24 \\
double[] & 8N+24 \\

\noalign{\smallskip}\hline\noalign{\smallskip}
\caption{for one-dimensional arrays}
\end{tabular}

\begin{tabular}{ll}
\hline\noalign{\smallskip}
\textbf{Type} & \textbf{Bytes} \\
\noalign{\smallskip}\hline\noalign{\smallskip}

char[][] & 2MN \\
int[][] & 4MN \\
double[][] & 8MN \\

\noalign{\smallskip}\hline\noalign{\smallskip}
\caption{for two-dimensional arrays}
\end{tabular}

\begin{tabular}{ll}
\hline\noalign{\smallskip}
\textbf{Type} & \textbf{Bytes} \\
\noalign{\smallskip}\hline\noalign{\smallskip}

Object overhead & 16 \\
Reference & 8 \\
Padding & 8x \\

\noalign{\smallskip}\hline\noalign{\smallskip}
\caption{for objects}
\end{tabular}

Reference includes object reference and innner class reference.

Padding is to make the object memory size of 8's multiple.
\subsection{Example}
The generics is passed as Boolean:
\begin{java}
public class Box<T> {   // 16 (object overhead)
    private in N;       // 4 (int)
    private T[] items;  // 8 (reference to array)
                        // 8N+24 (array of Boolean references)
                        // 24N (Boolean objects)
                        // 4 (padding to round up to a multiple)
}
\end{java} 

Notice the multiple levels of references. 
